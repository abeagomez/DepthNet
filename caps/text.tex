\documentclass{article}

\usepackage[utf8]{inputenc}
\usepackage[spanish]{babel}

\begin{document}

\section{Detección de Luz}\label{detecciuxf3n-de-luz}

Para la detección de luz se decidió realizar dos tipos de prueba:
detección de una luz blanca y detección de una luz infrarroja. Esto
permitió evaluar ambos resultados y elegir el de mejor desempeño.
(imágenes con luz blanca y con luz infrarroja)

En ambos casos, se utilizó la cámara RGB del Kinect para el
procesamiento de la imagen descrito a continuación.

Para comenzar, se aplicó a cada cuadro (frame) un ``Desenfoque
Gaussiano'' (Gaussian Blur) AGREGAR REFERENCIA. Esta técnica es
ampliamente utilizada en la visión por computadoras para reducir el
ruido y el nivel de detalle de una imagen. AQUI IMAGENES DE GAUSSIAN
BLUR.

La elección del ``Gaussian Blur'' como técnica para reducir el ruido de
la imagen estuvo basada en un estudio de ruido{[}2.6.3 de Computer
Vision Fundamentals{]} de la cámara del Kinect. El ruido fue medido a
partir de una porción de la imagen con una distribución de color
uniforme. Al analizar el histograma de distribución de intensidad de los
pixeles en dicha porción de imagen, se puede ver (IMAGEN DEBAJO) que la
distribución resultante es muy similar a la gaussiana, es por esto que
para eliminarlo se utilizó un filtro lineal gaussiano.

Luego, se aplicó un umbral binario (binary threshold) a la imagen
resultante, previamente transformada a escala de grises. Aplicar un
umbral binario a partir de determinado valor de intensidad permite
filtrar aquellas partes y objetos en la imagen que no serán de utilidad.
En el presente trabajo, se deseaba filtrar todo elemento de la imagen
que no se correspondiera con una zona de emisión de luz. Luego de
transformar la imagen a escala de grises, se pudo comprobar que los
emisores de luz son puntos de alta intensidad, cercanos al blanco; esto
permitió establecer un umbral bastante alto y filtrar la mayor parte de
los elementos de la imagen que no eran de interés. El proceso de
filtrado consistió en disminuir a 0 la intensidad de todos aquellos
pixeles que estuvieran por debajo del umbral establecido y aumentar a
255 la intensidad de los pixeles que superaran dicho umbral.

IMAGENES DE BINARY THRESHOLD

Al aplicar el proceso anterior utilizando un emisor de luz blanca,
surgió el problema de las superficies reflexivas: con frecuencia, el
ambiente de trabajo contiene un número elevado de superficies
reflexivas, que trae como resultado la detección de varios puntos de
emisión de luz en la imagen.

Otro de los problemas enfrentados fue la existencia de emisores de luz
blanca secundarios, que pueden ser fácilmente confundidos con el emisor
principal.

IMAGENES DEL EMISOR CON REFLEXION Y CON BOMBILLO DETRAS

Fue utilizado entonces un emisor de luz infrarroja. Se construyó un
filtro infrarrojo A BASE DE NEGATIVOS para la cámara del kinect, lo cual
permitió filtrar aquellos elementos de la imagen que no fueran emisores
de luz infrarroja. En ambientes de trabajo comunes se hace menos
frecuente la existencia de múltiples puntos de emisión de este tipo de
luz. Para todos los casos probados, obtuvimos como resultado únicamente
nuestro emisor.

Para preveer resultados múltiples como consecuencia de superficies
reflexivas aplicamos también aquí un umbral binario. Al tener nuestro
emisor de luz infrarroja considerablemente menos intensidad que nuestro
emisor de luz blanca, cualquier lectura adicional obtenida a causa de
una superficie reflexiva debería quedar por debajo del umbral binario
seleccionado.

IMAGENES CON FILTRO INFRARROJO

Finalmente, se seleccionó el emisor de luz infrarroja como la mejor
alternativa de solución para la detección de un punto único de emisión
de luz. Como consecuencia, la solución presentada es válida solo en
aquellos ambientes donde existe un único emisor de luz infrarroja.


\end{document}

